\documentclass{svmult}

\usepackage{mathtools}
\usepackage{newtxtext}
\usepackage{newtxmath}
\usepackage{tikz-cd}

% For mathcal fonts
\usepackage{eucal} 

\let\remark\relax
\let\theorem\relax
\let\lemma\relax
\let\definition\relax
\let\proposition\relax
\let\corollary\relax
\let\exercise\relax
\let\example\relax
\let\conjecture\relax

\let\openbox\relax

\def\factname{Fact}

\spnewtheorem{theorem}{\theoremname}[section]{\bfseries}{\itshape}
%\renewcommand\thetheorem{\arabic{theorem}}
\spnewtheorem{lemma}[theorem]{\lemmaname}{\bfseries}{\itshape}
\spnewtheorem{definition}[theorem]{\definitionname}{\bfseries}{\itshape}
\spnewtheorem{proposition}[theorem]{\propositionname}{\bfseries}{\itshape}
\spnewtheorem{corollary}[theorem]{\corollaryname}{\bfseries}{\itshape}
\spnewtheorem{exercise}[theorem]{\exercisename}{\bfseries}{\upshape}
\spnewtheorem{example}[theorem]{\examplename}{\bfseries}{\upshape}
\spnewtheorem{examples}[theorem]{\examplesname}{\bfseries}{\upshape}
\spnewtheorem{remark}[theorem]{\remarkname}{}{\upshape}
\spnewtheorem{conjecture}[theorem]{\conjecturename}{\bfseries}{\upshape}
\spnewtheorem{notation}[theorem]{\notationname}{\bfseries}{\upshape}
\spnewtheorem{convention}[theorem]{\conventionname}{\bfseries}{\upshape}
\spnewtheorem{fact}[theorem]{\factname}{\bfseries}{\upshape}


\newcommand{\Z}{\mathbb{Z}}
\newcommand{\id}{\mathrm{id}}
\newcommand{\im}{\operatorname{img}}
\newcommand{\Obj}{\operatorname{Obj}}
\newcommand{\Sym}{\operatorname{Sym}}
\newcommand{\End}{\operatorname{End}}

\newcommand{\category}[1]{{\normalfont\textbf{#1}}}
\newcommand{\Grp}{\category{Grp}}
\newcommand{\Hp}{\category{Hp}}

% para enumerar
\renewcommand{\labelenumi}{\arabic{enumi})}

% for QED
\let\proof\relax\let\endproof\relax
\usepackage{amsthm}

\makeindex

\begin{document}

\title*{Trusses: between braces and rings}
\author{Tomasz Brzezi\'nski}

\institute{Tomasz Brzezi\'nski \at Swansea University (UK) and University of Bialystok (Poland), 
\email{t.brzezinski@swansea.ac.uk}}

%\author{Leandro Vendramin}
%\institute{Leandro Vendramin \at Vrije Universiteit Brussel, \email{Leandro.Vendramin@vub.be}}

\maketitle

\abstract{
An algebraic structure is a collection of sets with operations. Typical and most widespread across mathematics are systems such as a semigroup, monoid, group, ring, field, associative algebra, vector space or module. In this  course we will study some simple algebraic systems which have recently gained prominent position in algebra and topology such as braces,  racks or quandles (sets with two operations interacting with  each other in prescribed ways).  In particular we will explore a little known fact (first described nearly 100 years ago by Pruefer and Baer) that one can give a definition of a group without requesting existence of the neutral element and inverses by using a ternary rather than a binary operation (i.e. an operation with three rather than the usual two inputs). A set with such a suitable ternary operation is known as a heap. By picking an element in a heap, the ternary operation is reduced to the binary group operation, for which the chosen element is the neutral element (the resulting group is known as  a retract). We will study properties and examples of heaps and relate them to the properties of corresponding groups (retracts). Next we will look at heaps with an additional binary operation that distributes over the ternary heap operation, known as trusses, relate them to both rings and braces, and study their properties and applications.
}

%{\let\clearpage\relax\tableofcontents}

\section{Introduction}

The notes correspond to a 8-hours mini-course 
taught by Prof. Tomasz Brzezi\'nski at the 
Department of Mathematics at ULB, Brussels, in May 2022.
The course is addressed to Master and PhD students,
researchers and any one else with an 
interest in (new) algebraic structures. 
Notes by
Leandro Vendramin. Please send comments and corrections to: \url{Leandro.Vendramin@vub.be} 

\section{Algebraic structures}

We start with some basic definitions from universal algebra. We refer
to \cite{MR620952} for more details. 

\begin{definition}
\index{Algebraic structure}
    An \textbf{algebraic structure} is a set $A$ 
    with a collection of maps (called operations) 
    $\alpha_i\colon A^{|\alpha_i|}\to A$ for $i\in I$. By 
    convention, $A^0=\{*\}$. This algebraic structure on $A$ 
    will be denoted by $(A,(\alpha_i)_{i\in I})$. 
    The number $|\alpha_i|\in\Z_{\geq0}$ is the
    \textbf{arity} of the operation $\alpha_i$. 
\end{definition}

For example, let $A$ be a set and 
$\alpha\colon A^{|\alpha|}\to A$ be a map. If $|\alpha|=0$, 
then $\alpha$ is a nullary operation. If $|\alpha|=1$, then 
$\alpha$ is a unary operation. If $|\alpha|=2$, then $\alpha$ is a binary operation. 

\begin{example}
    A semigroup is a set $A$ with an  
    operation $A\times A\to A$, $(a,b)\mapsto ab$. Thus 
    a semigroup 
    is an algebraic structure.
\end{example}

Other examples of algebraic structures are 
monoids, groups, vector spaces. 

\begin{definition}
\index{Algebraic structures!of the same type}
    We say that the algebraic structures $(A,(\alpha_i)_{i\in I})$ and $(B,(\beta_i)_{i\in I})$ 
    have the \textbf{same type} if $|\alpha_i|=|\beta_i|$ for all $i\in I$.
\end{definition}

\begin{definition}
\index{Homomorphism}
    Let $(A,(\alpha_i)_{i\in I})$ and $(B,(\beta_i)_{i\in I})$ be 
    algebraic structures of the same type. A map $f\colon A\to B$ 
    is a \textbf{homomorphism} of algebraic structures if 
    for every $i\in I$ the diagram 
    \[
    \begin{tikzcd}
	{A^{|\alpha_i|}} & A \\
	{B^{|\beta_i|}} & B
	\arrow["{\alpha_i}", from=1-1, to=1-2]
	\arrow["f", from=1-2, to=2-2]
	\arrow["{f^{|\alpha_i|}}"', from=1-1, to=2-1]
	\arrow["{\beta_i}"', from=2-1, to=2-2]
    \end{tikzcd}
    \]
    is commutative.     
\end{definition}

If $f\colon A\to B$ is a map and $X\subseteq A$ is a subset, 
we write $f|_X$ to denote 
the restriction of $f$ on $X$, that is the
map $f\colon X\to B$, $x\mapsto f(x)$. 

\begin{exercise}
    If $f\colon (A,(\alpha_i)_{i\in I})\to (B,(\beta_i)_{i\in I})$ is
    a homomorphism, then the \textbf{image}  
    $\left(\im(f),(\beta_i|_{\im(f)})_{i\in I}\right)$ of $f$ 
    is an algebraic structure of the same type. 
\end{exercise}

\begin{definition}
\index{Congruence}
    A \textbf{congruence} on 
    $(A,(\alpha_i)_{i\in I})$ is an equivalence relation $R$ on $A$ 
    such that for every $i\in I$ one has 
    \[
    a_k R b_k\quad\forall k\in\{1,\dots,|\alpha_i|\}\implies \alpha_i(a_1,\dots,a_{|\alpha_i|}) R \alpha_i(b_1,\dots,b_{|\alpha_i|}).
    \]
\end{definition}

\begin{exercise}
Prove the following statements:
\begin{enumerate}
    \item If $R$ is a congruence on $(A,(\alpha_i)_{i\in I})$, then 
    $\left(A/R,(\overline{\alpha_i})_{i\in I}\right)$, 
    where 
    \[
    \overline{\alpha_i}(\overline{a_1},\dots,\overline{a_{|\alpha_i|}})
    =\overline{\alpha_i(a_1,\dots,a_{|\alpha_i|})},
    \]
    is an algebraic structure of the same type. 
    \item If $f\colon (A,(\alpha_i)_{i\in I})\to (B,(\beta_i)_{i\in I})$ is a homomorphism, 
    then 
    \[
    a(\ker f)b\Longleftrightarrow f(a)=f(b)
    \]
    is a congruence. This is known as the \textbf{kernel relation}
    $\ker f$.  
    \item Every congruence is a kernel relation. 
\end{enumerate}
\end{exercise}

\section{Heaps}

\begin{definition}
\index{Heap}
A \textbf{heap} is a set $H$ with a ternary operation $H\times H\times H\to H$, $(x,y,z)\mapsto [x,y,z]$, 
such that for all $a,b,c,d,e\in H$, 
\begin{align}
    \label{eq:associativity}&[[a,b,c],d,e]=[a,b,[c,d,e]],\\
    \label{eq:Malcev}&[a,a,b]=[b,a,a]=b.
\end{align}
\end{definition}

Equality \eqref{eq:Malcev} is known as Malcev's identity. 
%An application of \eqref{eq:Malcev} yields the following result. 

\begin{definition}
\index{Heap!abelian}
    A heap $H$ is \textbf{abelian} if $[a,b,c]=[c,b,a]$ for all $a,b,c\in H$. 
\end{definition}

Homomorphism of heaps are defined in the usual way. Heaps and heap homomorphism for a category. 
It will be denoted by $\Hp$. 

The empty set $\emptyset$ is a heap. However, we will only work with non-empty heaps. 

\begin{example}
\label{exa:H(G)}
    If $G$ is a group, then $[a,b,c]=ab^{-1}c$ turns $G$ into a heap $H(G)$. Note that
    the heap $G$ is abelian if and only if $H(G)$ is abelian. Moreover, 
    if $f\colon G\to G_1$ is a homomorphism of groups, then $H(f)\colon H(G)\to H(G_1)$, $x\mapsto f(x)$, is a 
    homomorphism of heaps. 
\end{example}

\begin{example}
    Let $G$ be a group and $H$ be a subgroup of $G$. For every $a\in G$, the coset
    $aH=\{ah:h\in H\}$ is a heap with
    $[ax,ay,az]=axy^{-1}z$. 
\end{example}

Recall that a \textbf{groupoid} is a small category $\mathcal{C}$ in which every morphism is an isomorphism. 
We write $\Obj(\mathcal{C})$ to denote the set of objects of $\mathcal{C}$. If $A,B\in\Obj(\mathcal{C})$, 
then $\mathcal{C}(A,B)$ will be the set of morphisms $A\to B$. 

\begin{example}
    Let $\mathcal{C}$ be a groupoid and $A,B\in\Obj(\mathcal{C})$. Then $\mathcal{C}(A,B)$ is a heap with
    $[f,g,h]=f\circ g^{-1}\circ h$. Note that this is well-defined, as
    \[
    \begin{tikzcd}
	A & B & A & B
	\arrow["h", from=1-1, to=1-2]
	\arrow["{g^{-1}}", shift left=1, from=1-2, to=1-3]
	\arrow["f", from=1-3, to=1-4]
	\arrow["g", shift left=1, from=1-3, to=1-2]
    \end{tikzcd}
    \]
\end{example}

An \textbf{affine space} (over a field $F$) 
is a set $A$ with a free and transitive action of an $F$-vector space $\overrightarrow{A}$. This means
that there is a map $A\times\overrightarrow{A}\to A$, $(a,v)\mapsto a+v$, such that 
the following condition hold:
\begin{enumerate}
    \item $a+(v+w)=(a+v)+w$ for all $a\in A$ and $v,w\in\overrightarrow{A}$.
    \item $a+0=a$ for all $a\in A$.
    \item For every $a,b\in A$ there exists a unique $\overrightarrow{ab}\in\overrightarrow{A}$ such that
    $b=a+\overrightarrow{ab}$.
\end{enumerate}

\begin{example}
    Let $A$ be an affine space. Then $A$ is an abelian 
    heap with the operation $[a,b,c]=a+\overrightarrow{bc}$. 
    
    We first prove Malcev's identities:
    Clearly, $[a,a,b]=a+\overrightarrow{ab}=b$, as $\overrightarrow{ab}$ is the unique
    vector that sends $a$ to $b$. Similarly, $[b,a,a]=b$, as $\overrightarrow{aa}=0$. 
    
    We claim that
    $\overrightarrow{a(b+v)}=\overrightarrow{ab}+v$ for all $a,b\in A$ and $v\in\overrightarrow{A}$. 
    In fact, using Malcev's identities, 
    \[
    a+\overrightarrow{a(b+v)}=b+v=\left(a+\overrightarrow{ab}\right)+v=a+\left(\overrightarrow{ab}+v\right).
    \]
    Now we compute
    \begin{align*}
        [a,b,[c,x,y]] &= \left[a,b,c+\overrightarrow{xy}\right]
        =a+\overrightarrow{b\left(x+\overrightarrow{xy}\right)}\\
        &=a+\left(\overrightarrow{bc}+\overrightarrow{xy}\right)
        =\left(a+\overrightarrow{bc}\right)+\overrightarrow{xy}
        =[[a,b,c],x,y].
    \end{align*}
    To prove that the heap is abelian we first note that since 
    \[
    a+\left(\overrightarrow{ab}+\overrightarrow{ba}\right)=b+\overrightarrow{ba}=a,
    \]
    it follows that $\overrightarrow{ab}+\overrightarrow{ba}=0$. Since
    \[
    c=b+\overrightarrow{bc}=\left(a+\overrightarrow{ab}\right)+\overrightarrow{bc},
    \]
    it follows that 
    \[
    [c,b,a]=c+\overrightarrow{ba}=a+\overrightarrow{ab}+\overrightarrow{bc}+\overrightarrow{ba}
    =a+\overrightarrow{bc}
    =[a,b,c].
    \]
\end{example}

We now summarize the relationship between groups and heaps. Let $\Grp$ denote 
the category of groups and group homomorphism. 

\begin{theorem}\
\label{thm:heaps_and_groups}
\begin{enumerate}
    \item The assignment $G\to H(G)$ and $f\mapsto H(f)$ is a functor from $\Grp$ to $\Hp$.
    \item For any heap $H$ and any $e\in H$, the operation $ab=[a,e,b]$ turns $H$ into a group. 
    This group is known as the \textbf{retract} 
    $G(H,e)$ of $H$ at $e$. 
    \item If $f\colon H\to H'$ is a heap homomorphism, then for all $e\in H$ and $e'\in H'$
    the maps
    \begin{align*}
        &f_e^{e'}\colon G(H,e)\to G(H',e'),&&a\mapsto [f(a),f(e),e'],\\
        &f_{e'}^{e}\colon G(H,e)\to G(H',e'),&&a\mapsto [e',f(e),f(a)],
    \end{align*}
    are group homomorphisms. 
    \item If $H$ is a heap and $e\in H$, then $H(G(H,e))=H$.
    \item If $G$ is a group and $x\in G$, then $G(H(G),x)\simeq G$. 
\end{enumerate}
\end{theorem}

\begin{proof}[Sketch of the proof]
    Routine calculations prove 1), see Exercise \ref{exa:H(G)}. 
    
    Let us prove 2). By using \eqref{eq:Malcev} we obtain that $e$ is the identity of $G(H,e)$. For example,
    $ae=[a,e,e]=a$. If $a\in H$, then $a^{-1}=[e,a,e]$. In fact,
    \[
    aa^{-1}=[a,e,a^{-1}]=[a,e,[e,a,e]]=[[a,e,e],a,e]=[a,a,e]=e.
    \]
    The associativity is left as an exercise. 
    
    3) Let us prove that $f_e^{e_1}$ is a group homomorphism. Let $a,b\in H(G)$. On the one hand,
    \begin{align*}
    f_e^{e_1}(ab)&=f_e^{e_1}([a,e,b])\\
    &=[f([a,e,b]),f(e),e_1]\\
    &=[[f(a),f(e),f(b)],f(e),e_1]\\
    &=[f(a),f(e),[f(b),f(e),e_1]].
    \end{align*}
    On the other hand, 
    \begin{align*}
    f_e^{e_1}(a)f_e^{e_1}(b)&=[f(a),f(e),e_1][f(b),f(e),e_1]\\
    &=[[f(a),f(e),e_1],e_1,[f(b),f(e),e_1]]\\
    &=[f(a),f(e),[e_1,e_1,[f(b),f(e),e_1]]\\
    &=[f(a),f(e),[f(b),f(e),e_1]].
    \end{align*}
    The other equality is similar. 
    
    We prove 4). We start with a heap $H$. 
    Fix $e\in H$ and construct the group $G(H,e)$ with
    multiplication $(x,y)\mapsto xy=[x,e,y]$. Now we construct the heap
    $H(G(H,e))$ with operation $(a,b,c)\mapsto ab^{-1}c$. Recall that 
    $b^{-1}=[e,b,e]$. Thus 
    \begin{align*}
        ab^{-1}c 
        &= [ab^{-1}, e, c]
        = [[a,e,b^{-1}],e,c]\\
        &= [[a,e,[e,b,e]],e,c]
        = [[[a,e,e],b,e],e,c]\\
        &= [[a,b,e],e,c]
        = [a,b,[e,e,c]]
        = [a,b,c].
    \end{align*}
    
    To prove 5) recall that $G$ has multiplication $(a,b)\mapsto ab$. 
    Then $H(G)$ is a heap with $[a,b,c]=ab^{-1}c$ and for $x\in G$, 
    $G(H(G),x)$ is a group with
    multiplication $(a,b)\mapsto a\cdot b=[a,x,b]=ax^{-1}b$. The 
    map $f\colon G\to G(H(G),x)$, $a\mapsto ax$ is a group homomorphism, as 
    \[
    f(ab)=(ab)x=(ax)x^{-1}(bx)=f(a)x^{-1}f(b)=f(a)\cdot f(b).
    \]
    Moreover, $f$ is bijective with inverse $G(H(G),x)\to G$, $a\mapsto ax^{-1}$. 
\end{proof}

As an application of Theorem \ref{thm:heaps_and_groups} we can quickly prove
the several properties of heaps. However, we will 
present heap-theoretic proofs. 

For $n\in\Z_{\geq2}$ let 
$\Sym_n$ be the symmetric group in $n$ letters. 

\begin{theorem}
\label{thm:properties}
Let $H$ be a heap and $a,b,c,d,e\in H$. Then the following statements hold:
\begin{enumerate}
    \item If $[a,b,c]=d$, then $a=[d,c,b]$. 
    \item $[a,b,[c,d,e]]=[a,[d,c,b],e]$.
    \item In $[a,b,c]=d$, any three elements determine the fourth one. 
    \item $a=b$ if and only 
        if $[a,b,c]=c$ for all $c\in H$. 
\end{enumerate}
\end{theorem}

\begin{proof}
Let us start with 1). Using \eqref{eq:associativity} and \eqref{eq:Malcev}, 
\[
a=[a,b,b]=[a,b,[c,c,b]]=[[a,b,c],c,b]=[d,c,b].
\]

We now prove 2). Let $a,b,c,d,e\in H$. Then \eqref{eq:associativity} and \eqref{eq:Malcev} imply that
\begin{align*}
    a=[[a,b,c],c,b]=\left[[[a,b,c],d,d],c,b\right]=\left[[a,b,c],d,[d,c,b]\right].
\end{align*}
Using 1) and Malcev's identities we obtain that
\[
[a,b,c]=\left[a,[d,c,b],d\right]=\left[a,[d,c,b],[e,e,d]\right]=\left[[a,[d,c,b],e],e,d\right].
\]
Again by using 1) we conclude that
\[
[a,[d,c,b],e]=\left[[a,b,c],d,e\right].
\]

%Let $h\in H$. Then $G(H,h)$ is a group
%with $ab=[a,h,b]$ and $H(G(H,h))=H$. Moreover, 
%$xy^{-1}z=[x,y,z]$. On the one hand, 
%\begin{align*}
%    [a,b,[c,d,e]]=ab^{-1}[c,d,e]=ab^{-1}cd^{-1}e
%\end{align*}
%On the other hand,
%\begin{align*}
%    [a,[d,c,b],e]=a[d,c,b]^{-1}e=a(dc^{-1}b)^{-1}e=ab^{-1}cd^{-1}e.
%\end{align*}

Let us prove 3). Let $a,b,c\in H$ and $d=[a,b,c]$. In 1) 
we obtained that $a=[d,c,b]$. Similarly,  
\begin{align*}
\label{eq:Iknow3}
c=[b,b,c]=[[b,a,a],b,c]=[b,a,[a,b,c]]=[b,a,d].
\end{align*}
Now using 2) we obtain
\[
b=[b,a,a]=[c,c,[b,a,a]]=[c,[a,b,c],a]=[c,d,a].
\]

To prove 4) note that, if $a=b$, then $[a,b,c]=[a,a,c]=c$ for all $c\in H$ 
by Malcev's identity. Conversely,
if $[a,b,c]=c$ for all $c\in H$, then, 
in particular, $a=[a,b,b]=b$ by Malcev's identity. 
\end{proof}

\begin{exercise}
\label{xca:equality}
    Let $H$ be a heap and $a,b\in H$. Prove that
    $a=b$ if and only if there exists $c\in H$ such that 
    $[a,b,c]=c$.
\end{exercise}

\begin{exercise}
\label{xca:abelian}
    In an abelian heap,
    \[
    [x_1,y_1,x_2,y_2,\dots,x_n,y_n,x_{n+1}]=[x_{\sigma(1)},y_{\tau(1)},\dots,x_{\sigma(n)},y_{\tau(n)},x_{\sigma(n+1)}]
    \]
    for all $\sigma\in\Sym_{n+1}$ and $\tau\in\Sym_n$. 
\end{exercise}

The previous exercise and Malcev's identity \eqref{eq:Malcev}
give a useful trick that avoids painful calculations in the context of abelian heaps. 
Let us do a concrete example:
\begin{align*}
    [a,b,c,d,b]=[a,b,[c,d,b]]=[a,b,[b,d,c]]=[[a,b,b],d,c]=[a,d,c].
\end{align*}

\begin{definition}
    Let $H$ be a heap. 
    A non-empty subset $S$ of $H$ is a \textbf{subheap} if $[s,s_1,s_2]\in S$ for all 
    $s,s_1,s_2\in S$. 
\end{definition}

\begin{exercise}
    Let $H$ be a heap, $e\in H$ and $S$ be a non-empty subset of $H$. 
    Prove that $S$ is a subheap if and only if $S$ is a subgroup of $G(H,e)$. 
\end{exercise}

If $H$ is a heap and $S$ is a subheap of $H$, we define 
a \textbf{subheap relation} as follows:
\[
a\sim_S b\Longleftrightarrow [a,b,s]\in S\text{ for some $s\in S$}.
\]
Note that $a\sim_Sb$ if and only if $[a,b,s]\in S$ for all $s\in S$. 

\begin{proposition}
    Let $H$ be a heap and $S$ be a subheap. Then $\sim_S$ is an equivalence relation. 
\end{proposition}

\begin{proof}
    Let $a,b,c\in S$. Then $a\sim_S a$, as $[a,a,a]=a$ by Malcev's identity. 
    If $a\sim_S b$, then $[a,b,s]\in S$ for some $s\in S$. Thus, since 
    \[
    [b,a,[a,b,s]]=[[b,a,a],b,s]=[b,b,s]=s\in S,
    \]
    we obtain that $b\sim_S a$. 
    Finally, assume that $a\sim_S b$ and $b\sim_S c$. We know that 
    $c\sim_S a$, so 
    $[c,a,s]\in S$ for some $s\in S$. Thus 
    \[
    [a,c,[c,a,s]]=[[a,c,c],a,s]=[a,a,s]=s\in S
    \]
    and hence $a\sim_S c$. 
\end{proof}

Let $H$ be a heap and $S\subseteq H$ be a subheap. For $a\in H$ 
the orbit of $a$ is the set
\[
\overline{a}=\{c\in H:c\sim_S a\}
\]
We write $H/S$ to denote the set
of equivalence classes.

\begin{theorem}
    Let $H$ be a heap and $S$ be a subheap of $H$. The following statements hold:
    \begin{enumerate}
        \item For every $s\in S$, $\overline{s}=S$. 
        \item For every $a\in H$, $\overline{a}$ is a subheap of $H$.
        \item For every $a,b\in H$, 
        the map
        \[
        \tau_b^a\colon H\to H,\quad
        c\mapsto [c,b,a],
        \]
        is an automorphism of heaps. 
        \item For every $a,b\in H$, 
        $\overline{a}\simeq\overline{b}$ as heaps. 
        \item For every $a\in H$, $\sim_S=\sim_{\overline{a}}$. 
    \end{enumerate}
\end{theorem}

\begin{proof}
    Let us first prove 1). Let $s\in S$. To prove that 
    $\overline{s}\subseteq S$ let 
    $a\in H$ be such that $a\sim_Ss$. Then 
    $[a,s,t]=u\in S$ for some $t\in S$. By Theorem \ref{thm:properties}, 
    $a=[u,t,s]\in S$, as $S$ is a subheap. Conversely, 
    if $a\in S$, then, in particular, $[a,s,s]\in S$. Thus 
    $a\sim_Ss$. 

    We now prove 2). Let $x,y,z\in S$ be such that $x,y,z\in\overline{a}$. 
    Since $x\sim_Sa$ and $y\sim_Sa$, it follows that 
    $x\sim_Sy$, so there exists $s\in S$ such that $[x,y,s]\in S$.
    Since  
    \[
    [[x,y,z],z,s]=[x,y,[z,z,s]]=[x,y,s]\in S, 
    \]
    one has $[x,y,z]\sim_Sz$. Now the claim follows since 
    $z\sim_Sa$.
    
    Let us prove 3). On the one hand,
    \[
    \tau_b^a([x,y,z])=[[x,y,z],b,a]]=[x,y,[z,b,a]].
    \]
    On the other hand, using Theorem \ref{thm:heaps_and_groups}
    and Malcev's identities:
    \begin{align*}
        [\tau_b^a(x),\tau_b^a(y),\tau_b^a(z)] &= 
        [[x,b,a],[y,b,a],[z,b,a]]\\
        &=[x,b,[a,[y,b,a],[z,b,a]]\\
        &=[x,b,[a,[a,b,y],[z,b,a]]\\
        &=[x,b,[[a,a,b],[y,[z,b,a]]\\
        &=[[x,b,b],[y,[z,b,a]]\\
        &=[x,y,[z,b,a]].
    \end{align*}
    The map $\tau_b^a$ is bijective with inverse $\tau_a^b$, as, 
    for example, 
    \[
    \tau_b^a\tau_a^b(c)
    =\tau_b^a([c,a,b])=[[c,a,b],b,a]=[c,a,[b,b,a]]=[c,a,a]=c.
    \]
    
    4) Let $a,b\in H$. The map $\tau_a^b$ is an automorphism of heaps
    such that $\tau_a^b(a)=b$. We claim that 
    $\tau_a^b(\overline{a})=\overline{b}$. Let $x\in H$ be such that 
    $x\sim_Sa$. In particular, $a\sim_Sx$, so 
    $[a,x,s]\in S$ for some $s\in S$. Now 
    $\tau_a^b(x)\sim_Sb$, as 
    \begin{align*}
    \left[\tau_a^b(x),b,[a,x,s]\right]&=\left[[x,a,b],b,[a,x,s]\right]\\
    &=\left[x,a,[b,b,[a,x,s]\right]
    =\left[x,a,[a,x,s]\right]=s\in S.
    \end{align*}
    
    5) Assume first that $x\sim_{\overline{a}}y$. Then there exists $b\in\overline{a}$ 
    such that $[x,y,b]\sim_Sa$. Since $[x,y,b]\sim_Sa$ and $b\sim_Sa$, it follows
    that $[x,y,b]\sim_Sb$. Thus there exists $s\in S$
    such that 
    \[
    [x,y,s]=[x,y,[b,b,s]]=[[x,y,b],b,s]\in S.
    \]
    That is $x\sim_Sy$. Conversely, assume now that $x\sim_Sy$. Then 
    there exists $s\in S$ such that 
    \[
    [[x,y,b],b,s]=[x,y,[b,b,s]]=[x,y,s]\in S.
    \]
    Let $b\in\overline{a}$. Since $[x,y,b]\sim_Sb$, it follows that 
    $[x,y,b]\sim_Sa$ and hence $x\sim_{\overline{a}}y$. 
\end{proof}

\begin{definition}
    A subheap $S$ of $H$ is \textbf{normal} if 
    for every $a\in H$ and $s,e\in H$ one has 
    $[[a,e,s],a,e]\in S$.  
\end{definition}

Note a subheap $S$ of $H$ is normal if and only if 
for every $a\in H$ and $s,e\in S$ there exists 
$t\in S$ such that $[a,e,s]=[t,e,a]$.
    
\begin{exercise}
    Prove that a subheap $S$ of $H$ is normal if and only if 
    $S$ is a normal subgroup of $G(H,e)$ for every $e\in S$. 
\end{exercise}

\begin{theorem}
    Let $f\colon H\to K$ be a homomorphism of heaps. For $c\in H$
    let 
    \[
    \overline{c}=\{b\in H:f(b)=f(c)\}
    \]
    be the equivalence class of $c$ with respect to $\ker(f)$. 
    Then $\overline{c}$ is a subheap of $H$.  
\end{theorem}

\begin{proof}
    To prove that $\overline{c}$ is a subheap first note that 
    $c\in\overline{c}$, so $\overline{c}$ is non-empty. 
    Now let $x,y,z\in\overline{c}$. 
    Then $f(x)=f(y)=f(z)=$ 
    $[x,y,z]\in\overline{c}$, as
    \[
    f([x,y,z])=[f(x),f(y),f(z)]=[f(c),f(c),f(c)]=f(c).
    \]
    
    To prove that $\overline{c}$ is normal let $x,y\in\overline{c}$ and $a\in H$. 
    Since $f(x)=f(y)=f(c)$, 
    \begin{align*}
        f\left(\left[[x,y,s],x,y]\right]\right)&=\left[[f(a),f(x),f(y)],f(a),f(x)\right]\\ 
        &=\left[[f(a),f(c),f(c)],f(a),f(c)\right]\\
        &=[f(a),f(a),f(c)]=f(c).\qedhere
    \end{align*}
\end{proof}

\begin{definition}
    Let $f\colon H\to K$ be a homomorphism of heaps and $e\in f(H)$. 
    An $e$-kernel of $f$ is defined as the set
    \[
    \ker_ef=\{a\in H:f(a)=e\}.
    \]
\end{definition}

\begin{theorem}
    Let $f\colon H\to K$ be a homomorphism of heaps and $e\in f(H)$. 
    \begin{enumerate}
        \item $\ker_ef$ is a normal subheap of $H$. 
        \item $a\sim_{\ker_ef}b$ if and only if $f(a)=f(b)$. 
        \item For every $e,e_1\in f(H)$, $\ker_ef\simeq\ker_{e_1}f$. 
    \end{enumerate}
\end{theorem}

\begin{proof}
    Let us prove 1). By definition, $\ker_ef$ is non-empty. If $x,y,z\in\ker_ef$, then 
    $f(x)=f(y)=f(z)=e$. Since $f$ is a heap homomorphism, 
    \[
    f([x,y,z])=[f(x),f(y),f(z)]=[e,e,e]=e
    \]
    and hence $[x,y,z]\in\ker_ef$. To prove that $\ker_ef$ is normal
    note that
    \[
    f\left(\left[ [a,x,y],a,x\right]\right)
    =\left[ [f(a),f(x),f(y)],f(a),f(x)\right]
    =\left[ [f(a),e, e],f(a),e\right]=e.
    \]
    
    Now we prove 2). Let $S=\ker_ef$. Then 
    \begin{align*}
        a\sim_Sb &\Longleftrightarrow [a,b,s]\in S\text{ for some $s\in S$}\\
        & \Longleftrightarrow f([a,b,s])=e\text{ for some $s\in H$ such that $f(s)=e$}\\
        & \Longleftrightarrow [f(a),f(b),e]=e\\
        & \Longleftrightarrow f(a)=f(b),
    \end{align*}
    by Exercise \ref{xca:equality}. 
    
    Finally we prove 3). Let $x,y\in H$ be such that $f(x)=e$ and $f(y)=e_1$. 
    Note that
    \[
    \ker_ef=\{a\in H:f(a)=e\}=\{a\in H:f(a)=f(x)\}=\overline{x}
    \]
    Similarly, $\ker_{e_1}f=\overline{y}$. We know that both $\overline{x}$ and
    $\overline{y}$ are subheaps of $H$. They are isomorphic, 
    as $\tau_x^y(\overline{x})=\overline{y}$. In fact, if $z\in\overline{x}$, 
    then $\tau_x^y(z)\in\overline{y}$, as 
    \[
    f(\tau_x^y(z))=f([z,x,y])=[f(z),f(x),f(y)]=[f(x),f(x),f(y)]=f(y).\qedhere 
    \]
\end{proof}

We leave the proof of the following result as an exercise. 

\begin{corollary}
    Every congruence of heaps is a subheap relation with respect to a normal subheap.  
\end{corollary}

We now discuss universal differences between heaps and groups. 

Let $X$ be a non-empty set. The set $W(X)$ of reduced words on $X$ 
is defined as the set of words 
$x_1\cdots x_{2n+1}$ in elements of $X$ of odd length such that no
consecutive letters are the same, i.e. 
\[
W(X)=\{x_1\cdots x_{2n+1}:n\geq1,\,x_1,\dots,x_{2n+1}\in X,\,x_i\ne x_{i+1}\text{ for all $i$}\}
\]

For a word $x_1\cdots x_{2n+1}$ the opposite 
word is 
\[
(x_1\cdots x_{2n+1})^{\operatorname{t}}=x_{2n+1}\cdots x_2x_1.
\]
On $W(X)$ we define a ternary operation $[-,-,-]$ by \emph{grafting and pruning}. What does it mean? 
If $u,v,w\in W(X)$, then $[u,v,w]$ is obtained by removing (or pruning) all 
pairs of consecutive equal letters from $uv^{\operatorname{t}}w$ (obtained by
concatenation (or grafting) of $u$, $v^{\operatorname{t}}$ and $w$. One
proves that $\left(W(X),[-,-,-]\right)$ is a heap. It is the \textbf{free heap} on 
$X$ and it will be denoted by $\mathcal{H}(X)$. 

\begin{definition}
    Let $X$ be a set. A \textbf{free heap} on $\mathcal{H}(X)$ is a heap with with the following
    property: for any heap $H$ and any map $f\colon X\to H$ there exists a unique
    heap homomorphism $\varphi\colon \mathcal{H}(X)\to H$ such that
\[\begin{tikzcd}
	X & {\mathcal{H}(X)} \\
	H
	\arrow["\iota", from=1-1, to=1-2]
	\arrow["f"', from=1-1, to=2-1]
	\arrow["{\varphi }", dashed, from=1-2, to=2-1]
\end{tikzcd}\]
    commutes, where $\iota$ denotes the inclusion $X\to W(X)$. 
\end{definition}

\begin{example}
        $\mathcal{H}(\{x\})=\{x\}$.
\end{example}

\begin{example}
        $\mathcal{H}(\{x,y\})\simeq \mathcal{H}(\Z)$ via $x\mapsto 0$ and
        \[
        \underbrace{xyx\cdots yx}_{\text{$n$-times $y$}}\mapsto -n,
        \quad
        \underbrace{yx\cdots yx}_{\text{$n$-times $y$}}\mapsto n.
        \]
\end{example}

\begin{example}
    Let $X$ be a set. The \textbf{free abelian heap} $\mathcal{A}(X)$ 
    is defined as the free heap $\mathcal{H}(X)$ on $X$ modulo
    the action of 
    $\bigcup_{n}\Sym_{n+1}\times\Sym_n$ given by 
    \[
    (\sigma,\tau)\cdot x_1y_1x_2y_2\cdots x_ny_nx_{n+1}
    =x_{\sigma(1)}y_{\tau(1)}x_{\sigma(2)}y_{\tau(2)}\cdots x_{\sigma(n)}y_{\sigma(n)}x_{\sigma(n+1)}
    \]
    for $\sigma\in\Sym_{n+1}$ and $\tau\in\Sym_{n}$. 
\end{example}

We now describe the \textbf{direct sum} (or coproduct) $H_1\boxtimes H_2$ of the 
abelian heaps $H_1$ and $H_2$. Let 
$\iota_1\colon H_1\to H_1\boxtimes H_2$ and 
$\iota_2\colon H_2\to H_1\boxtimes H_2$ be the inclusions. 
It is precisely the coproduct of 
$H_1$ and $H_2$ in the category of abelian heaps. 
\[\begin{tikzcd}
	{H_1} & {H_1\boxtimes H_2} & {H_2} \\
	& K
	\arrow["{\iota_2}", from=1-2, to=1-3]
	\arrow[dashed, from=1-2, to=2-2]
	\arrow[from=1-3, to=2-2]
	\arrow["{\iota_1}", from=1-1, to=1-2]
	\arrow[from=1-1, to=2-2]
\end{tikzcd}\]

Let $U$ be the disjoiunt union of $H_1$ and $H_2$ and
$\mathcal{A}(U)$ be the free abelian heap on $U$...
\[
H_1\boxtimes H_2=\mathcal{H}\left(G(H_1,e_1)\oplus G(H_2,e_2)\oplus\Z\right)
\]

\section{Trusses}

\begin{definition}
    A left (resp. right) \textbf{skew truss} is a heap $(T,[-,-,-])$ 
    with an associative binary operation $(x,y)\mapsto xy$ 
    such that 
    \[
    a[b,c,d]=[ab,ac,ad]\qquad\text{(resp. $[a,b,c]d=[ad,bd,cd]$).}
    \]
    If $T$ is an abelian heap, then we drop skew from the terminology. A \textbf{truss}
    is a left and right truss. 
\end{definition}

\begin{example}
    The set $2\Z+1=\{2m+1:m\in\Z\}$ is a truss with
    the usual multiplication and $[x,y,z]=x-y+z$.
\end{example}

\begin{example}
    The set 
    \[
    \frac{2\Z+1}{2\Z+1}=\left\{\frac{a}{b}:a,b\text{ odd integers}\right\}
    \]
    is a truss with the usual multiplication
    and $[x,y,z]=x-y+z$. Note that both operations are well-defined. 
\end{example}

\begin{example}
    Let $H$ be an abelian heap. The set $\End(H)$ of heap endomorphisms $H\to H$
    is an abelian heap with 
    \[
    [f,g,h](a)=[f(a),g(a),h(a)].
    \]
    We prove that the operation is well-defined. Since $H$ is abelian, 
    we can use Exercise \ref{xca:abelian} to obtain that  
    \begin{align*}
        [f,g,h]\left([a,b,c]\right) &= \left[ f([a,b,c]), g([a,b,c]), h([a,b,c]) \right]\\
        &=\left[ [f(a),f(b),f(c)], [g(a),g(b),g(c)], [h(a),h(b),h(c)]\right]\\
        &=\left[ [f(a),g(a),h(a)], [f(b),g(b),h(b)], [f(c),g(c),h(c)]\right]\\
        &=\left[ [f,g,h](a), [f,g,h](b), [f,g,h](c)\right].
    \end{align*}
    Moreover, $\End(H)$ is a truss with the usual composition:
    \begin{align*}
    &\left([f,g,h]\circ k\right)(a)=[f(k(a)),g(k(a),h(k(a))]
    \shortintertext{and}
    &\left(f\circ [h,g,k]\right)(a)=f([g(a),h(a),k(a)])
    =[f(g(a)),f(h(a)),f(k(a))].
    \end{align*}
\end{example}

Let $F$ be a field. An associative $F$-algebra is an $F$-vector space $A$ 
with a bilinear associative multiplication. In particular, $A$ is a ring. What if $A$ is replaced by 
an affine space?

\begin{example}
    Let $F$ be a field...
    $F$ acts on $A$ by $(\lambda,a,b)\mapsto a+\lambda\overrightarrow{ab}$. An affine
    transformation is a pair $\left(f,\overrightarrow{f}\right)$, where 
    $f\colon A\to B$ is a map and $\overrightarrow{f}\colon \overrightarrow{A}\to\overrightarrow{B}$ is a linear
    transformation such that 
    \[
    \overrightarrow{f}\left(\overrightarrow{ab}\right)=\overrightarrow{f(a)f(b)}
    \]
    for all $a,b\in A$. This means that
    \[
        \overrightarrow{f}\left(\overrightarrow{ab}\right)
        =
    \]
    An affine space with an $F$-affine multiplication is a truss... 
\end{example}

A \textbf{skew ring} is a...

\begin{example}
    A (left) skew ring is a (left) truss
    with
    \[
    a[b,c,d]=a(b-c+d)=ab-ac+ad=[ab,ac,ad].
    \]
\end{example}

A (left) \textbf{skew brace} is a triple $(A,+,\circ)$, where $(A,+)$ and $(A,\circ)$ are
groups and $a\circ(b+c)=a\circ b-a+a\circ c$ holds for all $a,b,c\in A$. 

\begin{proposition}
    \begin{enumerate}
        \item Let $B$ be a skew brace. Then $\left(H(B),\circ\right)$ is a left skew truss.
        \item Any left skew truss...
    \end{enumerate}  
\end{proposition}

\begin{proof}
    Note that in a skew brace
    $x\circ(-y)=x-x\circ y+x$ for all $x,y$. Now 
    \begin{align*}
    a\circ [b,c,d]&=a\circ (b-c+d)\\
    &=a\circ b-a+a\circ(-c)-a+a\circ d\\
    &=a\circ b-a\circ c+a\circ d\\
    &=[a\circ b,a\circ c,a\circ d].
    \end{align*}
\end{proof}

\begin{example}
    A skew left truss is a skew...
    $a+b=[a,1,b]$. 
    $G(T,1)$ is a skew brace. 
\end{example}

\bibliographystyle{plain}
\bibliography{refs}

%\printindex

\end{document}
